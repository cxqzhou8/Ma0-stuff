\documentclass{article}
\usepackage{amsthm}
\usepackage{csquotes}
\usepackage{changepage}
\usepackage{amssymb}
\setlength{\parskip}{1em}

\begin{document}
\section*{Question 1a.}
\subsection*{Discussion}
\begin{itemize}
    \item We will show that $ f $ is injective.
    \item We will show that $ f $ is surjective.
\end{itemize}

\subsection*{Proof}
\begin{proof}
    To show that $ f $ is bijective, we must show that it is both injective and surjective.

    \noindent To prove that $ f $ is injective, let us first assume $ f(s_1) = f(s_2) $. Then, the statement can be rewritten as follows: $ ms_1 + b = ms_2 + b \Rightarrow ms_1 = ms_2 \Rightarrow s_1 = s_2 $.
    Thus, $ f $ is injective.

    \noindent Next, we will prove that $ f $ is surjective. Let $ t $ be an element in the co-domain of $ f $. Assume that there exists a pre-image in the domain $ s \in \mathbb{R} $. Then $ s $ must satisfy the equation
    $ f(s) = ms + b = t $. Notice that if $ s = \frac{-b + t}{m} $, the equation holds. Thus, every $ t \in \mathbb{R} $ has the pre-image $ \frac{-b + t}{m} \in \mathbb{R} $.

    \noindent Since we have shown $ f $ to be both injective and surjective, we can conclude that $ f $ is bijective.
\end{proof}

\section*{Question 1b.}
\noindent Since $ f $ is a bijection, it is invertible. The inverse $ f^{-1}(x) = \frac{1}{m}(x - b) $. We can confirm this by noting that
\[
    f^{-1}(f(x)) = \frac{1}{m}(mx + b - b) = \frac{1}{m}(mx) = x.
\]

\section*{Question 2.}
\subsection*{Discussion}
\begin{itemize}
    \item We will show that $ f $ is injective.
    \item We will show that $ f $ is surjective.
\end{itemize}

\subsection*{Proof}
\begin{proof}
    To show that $ f $ is bijective, we must show that it is both injective and surjective.

    \noindent First, we will prove that $ f $ is injective. Assume that $ f(s_1) = f(s_2) $. Notice that the statement can be rewritten as follows:
    \[
        \frac{\gamma s_1 + 1}{s_1 + \rho} = \frac{\gamma s_2 + 1}{s_2 + \rho} \Rightarrow (\gamma s_1 + 1)(s_2 + \rho) = (\gamma s_2 + 1)(s_1 + \rho) \Rightarrow
    \]
    \[
        \gamma s_1 s_2 + \gamma s_1 p + s_2 + \rho = \gamma s_1 s_2 + \gamma s_2 \rho + s_1 + \rho \Rightarrow \gamma s_1 \rho + s_2 = \gamma s_2 \rho + s_1 \Rightarrow
    \]
    \[
        s_2 - \gamma s_2 \rho = s_1 - \gamma s_1 \rho \Rightarrow  s_2 (1 - \gamma \rho) = s_1 (1 - \gamma \rho) \Rightarrow s_2 = s_1.
    \]
    Notice above that since $ \gamma \rho \neq 1 $, we are able to divide both sides by $ 1 - \gamma \rho $.
    
    \noindent Since when $ f(s_1) = f(s_2) $, $ s_1 = s_2 $, we can conclude that $ f $ is injective.

    \noindent Next, we will prove that $ f $ is bijective. Let $ t \in \mathbb{R}-\{\gamma\} $ be an element in the co-domain of $ f $. Assume that there exists a pre-image in the domain $ s \in \mathbb{R}-\{-\rho\} $.
    Then $ s $ must satisfy the equation $ f(s) = \frac{\gamma s + 1}{s + \rho} = t $. To obtain the solution, we will solve for $ s $:
    \[
        \frac{\gamma s + 1}{s + \rho} = t \Rightarrow \gamma s + 1 = t (s + \rho) \Rightarrow \gamma s + 1 = st + \rho t \Rightarrow \gamma s - st = \rho t - 1 \Rightarrow
    \]
    \[
        s (\gamma - t) = \rho t - 1 \Rightarrow s = \frac{\rho t - 1}{\gamma - t}.
    \]
    
    \noindent Notice that substituting $ s $ into $ f $ yields
    \[
        f(s) = \frac{\gamma \frac{\rho t - 1}{\gamma - t} + 1}{\frac{\rho t - 1}{\gamma - t} + \rho} = \frac{\frac{\gamma \rho t - \gamma}{\gamma - t} + \frac{\gamma - t}{\gamma - t}}{\frac{\rho t - 1}{\gamma - t} + \frac{\rho (\gamma - t)}{\gamma - t}} = \frac{\frac{t (\gamma \rho - 1)}{\gamma - t}}{\frac{\gamma \rho - 1}{\gamma - t}} = t.
    \]
    To verify that the pre-image $ \frac{\rho t - 1}{\gamma - t} \neq -\rho $, we will solve the following equation:
    \[
        \frac{\rho t - 1}{\gamma - t} = -\rho \Rightarrow \rho t - 1 = -\gamma \rho + \rho t \Rightarrow \gamma \rho = 1.
    \]
    Since $ \gamma \rho \neq 1 $, the pre-image will always be in the domain of $ f $.

    \noindent Thus, every $ t \in \mathbb{R}-\{\gamma\} $ has the pre-image $ \frac{\rho t - 1}{\gamma - t} \in \mathbb{R}-\{-\rho\} $.

    \noindent Since we have proven that $ f $ is both injective and surjective, we can conclude that $ f $ is a bijection.
\end{proof}

\section*{Question 3.}
\subsection*{Discussion}
\begin{itemize}
    \item \textbf{What we know:} Since $ g \circ f $ is injective, when $ g(f(s_1)) = g(f(s_2)) $, $ s_1 = s_2 $.
    \item We will show that if $ g \circ f $ is injective, then $ f $ is injective.
\end{itemize}

\subsection*{Proof}
\begin{proof}
    Let us consider the statement \textquote{If $ g \circ f $ is injective, then $ f $ is injective.} Assume that $ g \circ f $ is injective. Then, by the definition of injectivity, when $ g(f(s_1)) = g(f(s_2)) $, $ s_1 = s_2 $. Let us also assume, to the contrary,
    that $ f $ is not injective. Therefore, when $ f(s_1) = f(s_2) $, $ s_1 \neq s_2 $. However, notice that if $ s_1 \neq s_2 $ and $ f(s_1) = f(s_2) $, then $ g \circ f $ is not injective because it maps two different inputs $ s_1, s_2 $ to the same output --- a contradiction. Thus, $ f $ must be injective.
\end{proof}

\section*{Question 4a.}
\subsection*{Discussion}
\begin{itemize}
    \item \textbf{What we know:} Let $ a \in \mathbb{R} $ and $ f : \mathbb{R} \rightarrow \mathbb{R} $ given by $ f(x) = a $. Then 
    \[
        \int_{0}^{1} f(x) dx = \int_{0}^{1} a dx = ax \Bigr|_{0}^{1} = a - 0 = a.
    \]
    \item We will show that $ \varphi $ is surjective.
\end{itemize}

\subsection*{Proof}
\begin{proof}
    We will show that $ \varphi $ is surjective. Let $ a \in \mathbb{R} $ be an element in the co-domain of $ \varphi $. Assume that there exists a pre-image in the domain $ f \in C([0, 1]) $. Then $ f $ must satisfy the equation
    \[
        \varphi(f) = \int_{0}^{1} f(x) dx = a.
    \]
    Notice that if $ f(x) = a $, then the equation holds, as
    \[
        \int_{0}^{1} f(x) dx = \int_{0}^{1} a dx = ax \Bigr|_{0}^{1} = a - 0 = a.
    \]
    Also note that since $ f(x) = a $ is continuous for all $ x \in \mathbb{R} $, $ f \in C([0, 1]) $. Thus, every $ a \in \mathbb{R} $ has the pre-image $ f(x) = a \in C([0, 1]) $.
\end{proof}

\section*{Question 4b.}
\subsection*{Discussion}
\begin{itemize}
    \item \textbf{What we know:} Consider the element $ \frac{1}{3} $ in the co-domain of the function $ \varphi $. If $ f(x) = \frac{1}{3} $ and $ g(x) = x^2 $, then 
    \[
        \varphi(f) = \int_{0}^{1} f(x)dx = \frac{1}{3}x \Bigr|_{0}^{1} = \frac{1}{3} = \frac{1}{3} x^3 \Bigr|_{0}^{1} = \int_{0}^{1} x^2 dx = \varphi(g).
    \]
    \item We will show that the function $ \varphi $ is not injective.
\end{itemize} 

\subsection*{Proof}
\begin{proof}
    To prove that $ \varphi $ is not injective, let us first assume the contrary, that $ \varphi $ is injective. Let $ f(x) = \frac{1}{3} $ and $ g(x) = x^2 $ be functions in the domain of $ \varphi $. Observe that
    \[
        \varphi(f) = \int_{0}^{1} f(x) dx = \frac{1}{3}x \Bigr|_{0}^{1} = \frac{1}{3}.
    \]
    Also notice that
    \[
        \varphi(g) = \int_{0}^{1} g(x) dx = \frac{1}{3} x^3 \Bigr|_{0}^{1} = \frac{1}{3}.
    \]
    Since $ \varphi(f) = \varphi(g) $, but $ f \neq g $, $ \varphi $ cannot be injective.
\end{proof}
\end{document}