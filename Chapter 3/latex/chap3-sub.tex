\documentclass{article}
\usepackage{amsthm}
\usepackage{csquotes}
\usepackage{changepage}
\usepackage{amssymb}
\setlength{\parskip}{1em}

\begin{document}
\section*{Question 1a.}
\subsection*{Discussion}
\begin{itemize}
    \item We will show that $ f $ is injective.
    \item We will show that $ f $ is surjective.
\end{itemize}

\subsection*{Proof}
\begin{proof}
    To show that $ f $ is bijective, we must show that it is both injective and surjective.

    \noindent To prove that $ f $ is injective, let us first assume $ f(s_1) = f(s_2) $. Then, the statement can be rewritten as follows: $ ms_1 + b = ms_2 + b \Rightarrow ms_1 = ms_2 \Rightarrow s_1 = s_2 $.
    Thus, $ f $ is injective.

    \noindent Next, we will prove that $ f $ is surjective. Let $ t $ be an element in the co-domain of $ f $. Assume that there exists a pre-image in the domain $ s \in \mathbb{R} $. Then $ s $ must satisfy the equation
    $ f(s) = ms + b = t $. Notice that if $ s = \frac{-b + t}{m} $, the equation holds. Thus, every $ t \in \mathbb{R} $ has the pre-image $ \frac{-b + t}{m} \in \mathbb{R} $.

    \noindent Since we have shown $ f $ to be both injective and surjective, we can conclude that $ f $ is bijective.
\end{proof}

\section*{Question 1b.}


\end{document}