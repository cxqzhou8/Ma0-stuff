\documentclass{article}
\usepackage{amsthm}
\usepackage{csquotes}
\usepackage{changepage}
\setlength{\parskip}{1em}

\begin{document}
    \section*{Question 1.}
    \subsection*{Discussion}
    \begin{itemize}
        \item We will show there exists an $ x $ such that $ mx + b = 0 $.
        \item We will show that $ x $ is unique.
    \end{itemize}
    
    \subsection*{Proof}
    \begin{proof}
        Consider $ x = -\frac{b}{m} $. Note that $ mx + b = m(-\frac{b}{m}) + b = -b + b = 0 $. Thus, there exists a real $ x $ such that $ mx + b = 0 $.
        We will show that $ x $ is unique. Assume that both $ x $ and $ y $ satisfy $ mx + b = 0 $. Then $ mx + b = 0 = my + b $.
        Therefore, $ mx + b = my + b $, and thus $ mx = my $, and so $ x = y $. Thus, there exists a unique $ x $ such that $ mx + b = 0 $.
    \end{proof}

    \section*{Question 2.}
    \subsection*{Discussion}
    \begin{itemize}
        \item We will show that $ -1 \le x \le 1 \Rightarrow x^2 \le 1 $.
        \item We will show that $ x^2 \le 1 \Rightarrow -1 \le x \le 1 $.
    \end{itemize}
    
    \subsection*{Proof}
    \begin{proof}
        To prove this biconditional statement, we will prove two conditional statements.

        \noindent We will first show that the statement \textquote{If $ -1 \le x \le 1 $, then $ x^2 \le 1 $} is true. Assume that $ -1 \le x \le 1 $ is true.
        Then $ x \le 1 $ and $ x \ge -1 $. Since $ x \le 1 $, $ x \cdot x \le 1 \cdot 1 \Rightarrow x^2 \le 1 $, as $ x \ge -1 $. Thus, the original statement is true.

        \noindent Next we will prove the statement \textquote{If $ x^2 \le 1 $, then $ -1 \le x \le 1 $} by proving its contrapositve \textquote{If $ x < -1 $ or $ x > 1 $, then $ x^2 > 1 $.}
        Assume that $ x < -1 \vee x > 1 $ is true. Then only one of $ x < -1 $ or $ x > 1 $ needs to be true for $ x < -1 \vee x > 1 $ to be true. 
        Since $ x $ cannot be both greater than $ 1 $ and less than $ -1 $, only one of the statements in the compound statement can be true. 
        If $ x < -1 $ is true, $ x \cdot x < -1 \cdot -1 \Rightarrow x^2 > 1 $. If $ x > 1 $ is true, $ x \cdot x > 1 \cdot 1 \Rightarrow x^2 > 1 $.
        Thus, in either case the contrapositve will be true. Since we have proven the contrapositve, the original statement must also be true.

        \noindent Since we proved above the two conditional statements, the biconditional statement \textquote{Let $ x $ be a real number. $ -1 \le x \le 1 $ if and only
        if $ x^2 \le 1 $} is proven.
        
    \end{proof}

    \section*{Question 3.}
    \subsection*{Discussion}
    \begin{itemize}
        \item We will show that $ m $ and $ n $ have the same parity $ \Rightarrow m + n $ is even.
        \item We will show that $ m + n $ is even $ \Rightarrow m $ and $ n $ have the same parity.
    \end{itemize}

    \subsection*{Proof}
    \begin{proof}
        To prove this biconditional statement, we will prove two conditional statements.

        \noindent We will first show that the statement \textquote{If $ m $ and $ n $ have the same parity, then $ m + n $ is even} is true. Assume that $ m $ and $ n $ have the same parity. 
        Then there are two cases to consider:

        \noindent \textbf{Case 1. $ m $ and $ n $ are both even:} If $ m $ and $ n $ are both even, then they can be written as $ m = 2k $ and $ n = 2l $, for some whole numbers $ k $ and $ l $.
        Then $ m + n = 2k + 2l = 2(k + l) $. Since $ k + l $ is also a whole number, $ m + n $ is even.

        \noindent \textbf{Case 2. $ m $ and $ n $ are both odd:} If $ m $ and $ n $ are both odd, then they can be written as $ m = 2k + 1 $ and $ n = 2k + 1 $, for some whole numbers $ k $ and $ l $.
        Then $ m + n = 2k + 1 + 2l + 1 = 2k + 2l + 2 = 2(k + l + 1) $. Since $ k + l + 1 $ is also a whole number, $ m + n $ is even.
        
        \noindent Thus, the statement \textquote{If $ m $ and $ n $ have the same parity, then $ m + n $ is even} is proven.

        \noindent Next, we will prove the statement \textquote{If $ m + n $ is even, then $ m $ and $ n $ have the same parity.} We will instead prove its contrapositve \textquote{If $ m $ and $ n $ do not 
        have the same parity, then $ m + n $ is odd.} Assume that $ m $ and $ n $ do not have the same parity. Then let us consider the following two cases:

        \noindent \textbf{Case 1. $ m $ is even and $ n $ is odd:} If $ m $ is even and $ n $ is odd, then they can be written as $ m = 2k $ and $ n = 2l + 1 $, for some whole numbers $ k $ and $ l $. Then $ m + n = 2k + 2l + 1 = 2(k + l) + 1 $.
        Since $ k + l $ is also a whole number, $ m + n $ is odd.

        \noindent \textbf{Case 2. $ m $ is odd and $ n $ is even:} If $ m $ is odd and $ n $ is even, then they can be written as $ m = 2k + 1 $ and $ n = 2l $, for some whole numbers $ k $ and $ l $. Then $ m + n = 2k + 1 + 2l = 2(k + l) + 1 $.
        Since $ k + l $ is also a whole number, $ m + n $ is odd.

        \noindent Thus, the contrapositve \textquote{If $ m $ and $ n $ do not have the same parity, then $ m + n $ is odd} is true. Since we have proven the contrapositve, the original statement must also be true.

        \noindent Since we have proved above the two conditional statements, the biconditional statement \textquote{Let $ m $ and $ n $ be whole numbers. $ m $ and $ n $ have the same parity if and only if $ m + n $ is even} is proven.
    \end{proof}

    \section*{Question 4.}
    \subsection*{Discussion}
    \begin{itemize}
        \item We will show that $ m \cdot n $ is odd $ \Rightarrow m $ and $ n $ are both odd.
        \item We will do this by proving the contrapositve statement.
    \end{itemize}

    \subsection*{Proof}
    \begin{proof}
        We will show that the statement \textquote{Let $ m $ and $ n $ be whole numbers. If $ m \cdot n $ is odd, then $ m $ and $ n $ are both odd} is true. We will instead prove its contrapositve \textquote{If $ m $ and $ n $ are not both odd, then $ m \cdot n $ is even.}
        Assume that $ m $ and $ n $ are not both odd. Then, let us consider three cases:

        \noindent \textbf{Case 1. $ m $ is even and $ n $ is odd:} If $ m $ is even and $ n $ is odd, then they can be written as $ m = 2k $ and $ n = 2l + 1 $, for some whole numbers $ k $ and $ l $. Then $ m \cdot n = 2k(2l + 1) = 4kl + 2k = 2(2kl + k) $. Since $ 2kl + k $ is
        also a whole number, $ m \cdot n $ is even.

        \noindent \textbf{Case 2. $ m $ is odd and $ n $ is even:} If $ m $ is odd and $ n $ is even, then they can be written as $ m = 2k + 1 $ and $ n = 2l $, for some whole numbers $ k $ and $ l $. Then $ m \cdot n = (2k + 1)(2l) = 4kl + 2l = 2(2kl + l) $. Since $ 2kl + l $ is
        also a whole number, $ m \cdot n $ is even.

        \noindent \textbf{Case 3. $ m $ and $ n $ are both even:} If $ m $ and $ n $ are both even, then they can be written as $ m = 2k $ and $ n = 2l $, for some whole numbers $ k $ and $ l $. Then $ m \cdot n = (2k)(2l) = 4kl = 2(2kl) $. Since $ 2kl $ is
        also a whole number, $ m \cdot n $ is even.

        \noindent Thus, in all three cases the contrapositve statement \textquote{If $ m $ and $ n $ are not both odd, then $ m \cdot n $ is even} is true. Since we have proven the contrapositve, the original statement must also be true.
    \end{proof}


    \section*{Question 5.}

    \renewcommand{\labelenumi}{(\alph{enumi})}
    \renewcommand{\labelenumii}{\arabic{enumii}.}
    \begin{enumerate}
        \item For the odd whole numbers $ n = -3, -1, 1, 3, 5, 7, 9 $, write $ n $ as the difference of two perfect squares.
        \begin{enumerate}
            \item $ n = -3 $: $ 1^2 - 2^2 = -3 $
            \item $ n = -1 $: $ 0^2 - 1^2 = -1 $
            \item $ n = 1 $: $ 1^2 - 0^2 = 1 $
            \item $ n = 3 $: $ 2^2 - 1^2 = 3 $
            \item $ n = 5 $: $ 3^2 - 2^2 = 5 $
            \item $ n = 7 $: $ 4^2 - 3^2 = 7 $
            \item $ n = 9 $: $ 5^2 - 4^2 = 9 $
        \end{enumerate}
        \item \textbf{Proposition:} Every odd whole number can be written as the difference of two perfect squares.
    \end{enumerate}

    \begin{adjustwidth}{6ex}{}
        
        \textbf{Discussion}
        \begin{itemize}
            \item We will show that an odd whole number $ n $ can be written as $ n = 2k + 1 $, for some whole number k.
            \item We will show that $ (k + 1)^2 - k^2 = 2k + 1 $.
        \end{itemize}
        
        \noindent \textbf{Proof}
        \begin{proof}
            Let us consider the statement \textquote{Every odd whole number can be written as the difference of two perfect squares.} Notice that for any odd whole number $ n $, $ n $ can be written as $ n = 2k + 1 $, for some whole number $ k $. Notice also that
            the difference between the squares of $ k + 1 $ and $ k $ can be written as $ (k + 1)^2 - k^2 = k^2 + 2k + 1 - k^2 = 2k + 1 $. Since $ 2k + 1 = n $ too, we have proven the original statement.
        \end{proof}
    \end{adjustwidth}
\end{document}